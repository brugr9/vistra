\chapter{Project Management}
% 
\section{Time Management}
\label{sec:Time Management}
Das Modul BTI-7301 Projekt 1 startete mit Beginn des Herbstsemester 2013/14. In der Woche 38 wurden die Projekte durch die Dozenten vorgestellt. Es wurden Teams gebildet und den den Projekten bzw. den Dozenten zugeteilt. Ein erstes Treffen des zust\"andigen Dozenten mit dem Team fand statt und erste Vereinbarungen wurden getroffen. Der Zeitplan gliedert sich nun wie folgt in vier Phasen:
\begin{description}
  \item[Phase 1: Projektplanung und Systemarchitektur] Wochen 39/40/41(/42) 2013 (3-4 Wochen)
  \begin{itemize}
    \item Einarbeiten in die Thematik (Graphen, Algorithmen, \dots)
    \item Erstellen der Requirements, Spezifikation
    \item Design der Systemarchitektur
  \end{itemize}
  \item[Phase 2: W\"ochentliche Sprintzyklen] Wochen (42/)43 bis 51 2013 (9-10 Wochen)
  \begin{itemize}
    \item Use Case w\"ahlen f\"ur n\"achsten Sprint
    \item (Re-)Design der Interfaces und Klassenhierarchie, Test und Implementation
    \item Dokumentation
  \end{itemize}
  \item[Phase 3: Projektabschluss] Wochen 52 2013 und 01 2014 (2 Wochen)
  \begin{itemize}
    \item Dokumentation abschliessen
    \item Erstellen der Pr\"asentation
  \end{itemize}
  \item[Phase 4: Pr\"asentation des Projektes] Wochen 02 und 03 2014 (2 Wochen)
  \begin{itemize}
    \item Besprechen des Ablaufes der Pr\"asentation
    \item Checklisten Medien und Ger\"ate
    \item Pr\"asentation
  \end{itemize}
\end{description}
% 
\section{Object Oriented Analysis and Design}
\label{sec:Object Oriented Analysis and Design}
Die Entwicklung des Projektes erfolgt objektorientiert und wird laufend dokumentiert. Die Elaboration der Komponenten und der Aufbau der Dokumentation richten sich nach C. Larman~\cite{larmann:2004}. F\"ur die Formulierung der Use Cases im ausgearbeiteten Format wurde ein \LaTeX-Style-File~\cite{bruggmann:2013} erstellt und verwendet.
% 
\section{Development Environment Description}
\label{sec:Development Environment Description}
% 
\subsection{Programming Language and Libraries}
\label{subsec:Programming Language and Libraries}
Das System wird in der Programmiersprache Java implementiert. Es werden das Java Universal Network/Graph Framework JUNG~\cite{jung:2013}, das XML-basierte Format GraphML~\cite{graphml:2013}, die Generic Method Pattern von Roberto Tamassia, Michael Goodrich und Eric Zamor~\cite{tamassiagoodrichzamor:2013}, die Apache Commons IO Library und das Java Swing Framework verwendet. Als integrierte Entwicklungsumgebung kommt die Software Eclipse IDE zum Einsatz.
% 
\subsection{Sourcecode Management}
\label{subsec:Sourcecode Management}
F\"ur das Sourcecode Management (SCM) wird die Software git verwendet, das Projekt wird auf der webbasierten Plattform \textit{github} gehalten. In der Woche 45 wurde entschieden, das Projekt als Maven-Projekt zu halten. In der Woche 51 wurde entschieden, dass die Teammitglieder je ein Projekt alleine finalisieren.

\begin{itemize}
  \item Ab Woche 39: \url{https://github.com/brugr9/ch.bfh.bti7301.hs2013.GraphVisualisierung2}
  \item Ab Woche 45: \url{https://github.com/brugr9/gravis}
  \item Ab Woche 51: \url{https://github.com/brugr9/vistra}
\end{itemize}


1) Mein Projekt heisst nun vistra.

2) Ich habe den von Patrick Kofmel geschriebene Code
mehrheitlich gelöscht und nun selber neu implementiert
sowie Teile umgeschrieben (gui.view.component.viewer und sub packages), damits auf meine Implementation passt.

3) Ich habe auf meinem Server ein eigenes git-repository aufgesetzt.
- Gitweb:
http://scm.geogeek.ch/?p=vistra.git
- Klonen:
git clone http://scm.geogeek.ch/git/vistra.git

4) Dokumentation
Direktlink zum pdf:
http://scm.geogeek.ch/?p=vistra.git;a=blob_plain;f=GraphVisualisierung2/doc/vistra/ooad/bti7301-project1.pdf


Die Vorg\"anger-Projekte sind weiterhin bis mindestens nach Erhalt der Modul-Bewertung erreichbar.