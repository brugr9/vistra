\section{Projekt Management}
\label{sec:Projekt Management}
% 
\subsection{Time Management}
\label{subsec:Time Management}
Das Modul BTI-7301 Projekt 1 startete mit Beginn des Herbstsemester 2013/14. In der Woche 38 wurden die Projekte durch die Dozenten vorgestellt. Es wurden Teams gebildet und den den Projekten bzw. den Dozenten zugeteilt. Ein erstes Treffen des zust\"andigen Dozenten mit dem Team fand statt und erste Vereinbarungen wurden getroffen. Der Zeitplan gliedert sich nun wie folgt in vier Phasen:
\begin{description}
  \item[Phase 1: Projektplanung und Systemarchitektur] Wochen 39/40/41(/42) 2013 (3-4 Wochen)
  \begin{itemize}
    \item Einarbeiten in die Thematik (Graphen, Algorithmen, \dots)
    \item Erstellen der Requirements, Spezifikation
    \item Design der Systemarchitektur
  \end{itemize}
  \item[Phase 2: W\"ochentliche Sprintzyklen] Wochen (42/)43 bis 51 2013 (9-10 Wochen)
  \begin{itemize}
    \item Use Case w\"ahlen f\"ur n\"achsten Sprint
    \item (Re-)Design der Interfaces und Klassenhierarchie, Test und Implementation
    \item Review, Anpassung der Planung
  \end{itemize}
  \item[Phase 3: Projektabschluss] Wochen 52 2013 und 01 2014 (2 Wochen)
  \begin{itemize}
    \item Refactoring und Systemtests
    \item Erstellen der Pr\"asentation
  \end{itemize}
  \item[Phase 4: Pr\"asentation des Projektes] Wochen 02 und 03 2014 (2 Wochen)
  \begin{itemize}
    \item Besprechen des Ablaufes der Pr\"asentation
    \item Checklisten Medien und Ger\"ate
    \item Pr\"asentation
  \end{itemize}
\end{description}
% 
\subsection{Development Environment}
\label{subsec:Development Environment}
Das System wird in Java als Maven-Projekt implementiert. Es werden das Java Universal Network/Graph Framework (JUNG, \url{http://jung.sourceforge.net/}), das XML-basierte Format GraphML (\url{http://graphml.graphdrawing.org/}), die Apache Commons IO Library und das Java Swing Framework verwendet. F\"ur das Sourcecode Management (SCM) wird die Software git verwendet, das Projekt wird auf github gehalten (\url{https://github.com/brugr9/gravis}). Als IDE kommt Eclipse mit Egit zum Einsatz.